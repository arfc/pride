\section{Model and Data}

The system we modeled in this study is based on the current energy mix of
\gls{uiuc} which is summarized in Table \ref{tab:model}. \gls{temoa} optimizes
the years 2021-2030 in single year increments. Each year is divided 6 time
slices, three seasons, and a day-night cycle. The seasons are summer, winter,
and an ``inter'' season that represents spring and fall. The typical demand
breakdown for each season is determined by averaging historical data from 2015-
2018 \cite{marquissee_uiuc_2019}. \gls{app} is a natural gas and coal fired
cogeneration plant that fulfills all of the steam demand for \gls{uiuc} and
much of the electricity demand \cite{affiliated_engineers_inc_utilities_2015}.
In order to effectively capture cogeneration from \gls{app} in our model, we
introduced an intermediate technology, \texttt{TURBINE}. Thus, \gls{app} can
produce a steam ``commodity'' that is split between campus steam demand and
campus electricity demand via \texttt{TURBINE}. Introducing this intermediate
technology also allowed us to easily plug in new sources of steam to the
energy mix, like a nuclear reactor.

The capacities for solar and wind power are both capped in this \gls{temoa}
model and reflects the real constraints on the \gls{uiuc} energy mix.
The cap on solar energy is due to the maximum capacity of the solar
farms on campus. Currently, the solar farm is rated to produce 4.68 MWe, but
will be tripled in 2022 when the university finishes the planned
Solar Farm 2.0 \cite{noauthor_solar_nodate,white_solar_2017}.The cap on wind
energy is due to the 10-year power purchase agreement between UIUC and Rail
Splitter Wind Farm. This contract ends in 2026, at which point the university
can elect to purchase more or not \cite{breitweiser_wind_2016}. The current
installed capacity of Rail Splitter Wind Farm is 100.5 MW$_e$ so we limited
the maximum \gls{ppa} to 100.5 MW$_e$.

Carbon emissions are counted as a ``CO$_2$ equivalent'' which matches the
strategy adopted by \gls{icap}. The limits for each year are also based on
the \gls{icap} goals which were only published for three years: 2020, 2025, and
2050. We used linear interpolation to fill in the missing values. In our
\gls{temoa} model, we only tracked emissions from each unit of energy generated
rather than the lifetime carbon emissions.  

\begin{table*}[ht]
  \centering
  \caption{A summary of the technologies at \gls{uiuc}}
  \label{tab:model}
  \begin{tabular}{|cccccc|}
    \hline
    Technology & Name & Capital Cost & Variable Cost & Emissions& Citation\\
    & & M\$/MW & M\$/GWh & kton-CO$_2$eq/MW&\\
    \hline
    Natural Gas \& Coal Plant & \texttt{ABBOTT} & 0.735 & 0.0553 & 0.192&\cite{isee_illinois_2015,affiliated_engineers_inc_utilities_2015,us_department_of_energy_capital_2016,uiuc_fs_edna_nodate,energy_information_administration_electricity_nodate}\\
    Nuclear & \texttt{NUCLEAR} & 5.945 & 0.027 & - &\cite{us_department_of_energy_capital_2016,desai_nuclear_2018,wna_nuclear_2017,noauthor_eti_2018}\\
    Turbine & \texttt{TURBINE} & - & 0.03  & - &\cite{uiuc_fs_edna_nodate,affiliated_engineers_inc_utilities_2015}\\
    Photovoltaic Solar & \texttt{IMPSOL} & 1.66 & 0.196 & - &\cite{noauthor_solar_nodate,uiuc_fs_edna_nodate}\\
    Wind \gls{ppa} & \texttt{IMPWIND} & - & 0.0384 & - & \cite{breitweiser_wind_2016,uiuc_fs_edna_nodate}\\
    MISO Electricity Imports & \texttt{IMPELC} & - & 0.13 & 0.825&\cite{isee_illinois_2015,uiuc_fs_edna_nodate,noauthor_abbott_nodate}\\
    \hline
  \end{tabular}
\end{table*}

\begin{enumerate}
  \item Explain the time horizon
  \begin{itemize}
    \item 2020 is considered a historical year and reflects the current energy
    mix of the university.
    \item The model optimizes years 2021-2030 in single year increments.
    \item In this study, one year is  Future work will refine this temporal detail.
  \end{itemize}
  \item Typical demand for winter, summer, and the spring/fall ``inter" season
  are determined by averaging historical data from 2015-2018.
  \item The natural gas plant, "ABBOTT" as a cogeneration plant that produces
  all of the steam on campus and much of the electricity. In order to capture
  the cogeneration, we introduced a "TURBINE" technology that produces electricity
  from steam. The proposed nuclear reactor will also produce steam that "TURBINE"
  can use to produce electricity. Thus the model assumes that the nuclear reactor
  will serve as a direct replacement of "ABBOTT" or function alongside "ABBOTT"
  in an identical way.
  \item Natural gas prices.\\
  The price of natural gas is one of main factors driving the choice of energy
  production at UIUC. Since 2014, natural gas prices have somewhat steadily declined.
  \item Carbon Emissions \\
  Carbon emissions in the model are captured by using a carbon emission equivalent
  that matches the strategy adopted by iCAP.
  \item Capacity caps\\
  Solar and wind capacities are both capped by Temoa and reflect the reality of
  the UIUC energy mix.
  \begin{itemize}
    \item The cap on solar energy is due to the maximum capacity of the solar
    farms on campus. Currently, the solar farm is rated to produce 4.68 MWe, but
    will be quadrupled in 2022 when the university finishes the planned Solar Farm 2.0.
    \item The cap on wind energy is due to the 10-year power purchase agreement
    between UIUC and Rail-splitter Wind Farm. This contract ends in 2026, at
    which point the university can elect to purchase more or not.
  \end{itemize}
  \item Offsets, Growth, and Building Standards \\
  This model assumes an energy demand growth of 1\% per year. Thus, offsets like
  shutting down the Blue Waters Supercomputer and improving building standards,
  which serve to reduce demand, are not accounted for and assumes the university
  will carry on with business as usual in every regard except its energy mix.
  \item Scenarios \\
  Describe the modeled scenarios - BAU, 1, 2, 3. \\
  Uncertainty analysis is only performed on scenario 3 because scenarios 1 and 2
  will be pushed along the same technology trajectory because not limiting the
  size of the nuclear reactor means demand and emissions constraints can be
  satisfied arbitrarily. The business as usual scenario is not analyzed for
  uncertainty because it served as a sanity check to verify that Temoa was giving
  appropriate results.
\end{enumerate}
