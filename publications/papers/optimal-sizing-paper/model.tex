\section{Model and Data}

The system we modeled in this study is based on the current energy mix of
\gls{uiuc} which is summarized in Table \ref{tab:model}. \gls{temoa} optimizes
the years 2021-2030 in single year increments. Each year is divided 6 time
slices, three seasons, and a day-night cycle. The seasons are summer, winter,
and an ``inter'' season that represents spring and fall. The typical demand
breakdown for each season is determined by averaging historical data from 2015-
2018 \cite{marquissee_uiuc_2019}. \gls{app} is a natural gas and coal fired
cogeneration plant that fulfills all of the steam demand for \gls{uiuc} and
much of the electricity demand \cite{affiliated_engineers_inc_utilities_2015}.
In order to effectively capture cogeneration from \gls{app} in our model, we
introduced an intermediate technology, \texttt{TURBINE}. Thus, \gls{app} can
produce a steam ``commodity'' that is split between campus steam demand and
campus electricity demand via \texttt{TURBINE}. Introducing this intermediate
technology also allowed us to easily plug in new sources of steam to the
energy mix, like a nuclear reactor.
\begin{table*}[ht]
  \centering
  \caption{A summary of the technologies at \gls{uiuc}}
  \label{tab:model}
  \begin{tabular}{|cccccc|}
    \hline
    Technology & Name & Capital Cost & Variable Cost & Emissions& Citation\\
    & & M\$/MW & M\$/GWh & kton-CO$_2$eq/MW&\\
    \hline
    Natural Gas \& Coal Plant & \texttt{ABBOTT} & 0.735 & 0.0553 & 0.192&\cite{isee_illinois_2015,affiliated_engineers_inc_utilities_2015,us_department_of_energy_capital_2016,uiuc_fs_edna_nodate,energy_information_administration_electricity_nodate}\\
    Nuclear & \texttt{NUCLEAR} & 5.945 & 0.027 & - &\cite{us_department_of_energy_capital_2016,desai_nuclear_2018,wna_nuclear_2017,noauthor_eti_2018}\\
    Turbine & \texttt{TURBINE} & - & 0.03  & - &\cite{uiuc_fs_edna_nodate,affiliated_engineers_inc_utilities_2015}\\
    Photovoltaic Solar & \texttt{IMPSOL} & 1.66 & 0.196 & - &\cite{noauthor_solar_nodate,uiuc_fs_edna_nodate}\\
    Wind \gls{ppa} & \texttt{IMPWIND} & - & 0.0384 & - & \cite{breitweiser_wind_2016,uiuc_fs_edna_nodate}\\
    MISO Electricity Imports & \texttt{IMPELC} & - & 0.13 & 0.825&\cite{isee_illinois_2015,uiuc_fs_edna_nodate,noauthor_abbott_nodate}\\
    \hline
  \end{tabular}
\end{table*}

The capacities for solar and wind power are both capped in this \gls{temoa}
model and reflects the real constraints on the \gls{uiuc} energy mix.
The cap on solar energy is due to the maximum capacity of the solar
farms on campus. Currently, the solar farm is rated to produce 4.68 MWe, but
will be tripled in 2022 when the university finishes the planned
Solar Farm 2.0 \cite{noauthor_solar_nodate,white_solar_2017}.The cap on wind
energy is due to the 10-year power purchase agreement between UIUC and Rail
Splitter Wind Farm. This contract ends in 2026, at which point the university
can elect to purchase more or not \cite{breitweiser_wind_2016}. The current
installed capacity of Rail Splitter Wind Farm is 100.5 MW$_e$ so we limited
the maximum \gls{ppa} to 100.5 MW$_e$.

Carbon emissions are counted as a ``CO$_2$ equivalent'' which matches the
strategy adopted by \gls{icap}. The limits for each year are also based on
the \gls{icap} goals which were only published for three years: 2020, 2025, and
2050. We used linear interpolation to fill in the missing values. In our
\gls{temoa} model, we only tracked emissions from each unit of energy generated
rather than the lifetime carbon emissions. The reason for this is that
\gls{temoa} will give ``no solution'' if we include emissions from constructing
solar farms, nuclear reactors, or other energy sources because even if
\gls{uiuc} moved toward an energy mix that is based on a purely wind \gls{ppa},
the steam demand will be left unsatisfied.

This model assumes an energy demand growth of 1\% per year. Thus, offsets like
shutting down the Blue Waters Supercomputer, zero net-growth on campus, and
improving building standards that serve to reduce demand, are not accounted for
and assumes the university will carry on with business as usual in every regard
except its energy mix. We consider four scenarios, summarized in Table
\ref{tab:scenarios}. The first scenario maps out the emissions and energy
generation for \gls{uiuc} if the university continues with business as usual.
The numbered scenarios include nuclear capacity in the energy mix. Starting with
a ``free'' reactor in Scenario 1, then adding a capital cost in Scenario 2, and
finally, limiting the reactor capacity in Scenario 3. We also conducted an
MGA uncertainty analysis on Scenario 3 only because the first two scenarios will
be pushed along the same technology trajectories when faced with carbon
constraints.

\begin{table*}[ht]
  \centering
  \caption{Summary of \gls{temoa} Nuclear Scenarios}
  \label{tab:scenarios}
  \begin{tabular}{|ccccc|}
    \hline
    Scenario & Nuclear & Variable Cost & Capital Cost & Capacity Limit\\
    \hline
    BAU & No & - & - & - \\
    1 & Yes & Yes & - & - \\
    2 & Yes & Yes & Yes & - \\
    3 & Yes & Yes & Yes & 100 MW$_{th}$\\
    \hline
  \end{tabular}
\end{table*}
