\section{Conclusion}

In this study we used the \gls{esom} called \gls{temoa} to find the optimal
size of a nuclear reactor for the \gls{uiuc} microgrid. We first showed that
\gls{temoa} gave realistic results that matched predictions from both \gls{icap}
and the \gls{uiuc} Master Plan \cite{isee_illinois_2015, affiliated_engineers_inc_utilities_2015}.
Then we considered three scenarios that introduced nuclear capacity to
\gls{uiuc}. The first two scenarios did not constrain the size of the nuclear
reactor and thus satisfied the carbon constraints and exceeded the steam
and electricity demand requirements by building more nuclear capacity than
required. Importantly, \gls{temoa} has a tendency to over-build nuclear capacity
in the first two scenarios. Steam and electricity demands can be met on campus
by simply installing enough capacity to replace \gls{app} rather than exceed it
substantially.
The \gls{uiuc} Master Plan found that the goals outlined in \gls{icap} could
not be achieved with \gls{uiuc}'s current energy mix, which we corroborated in
our business-as-usual scenario. We showed in Scenario 3 that the \gls{icap}
goals could be met for the next decade by adding a modest capacity for nuclear
energy production. The assumptions of the model used in this study include
contributions from renewables, but exclude requirements of zero growth,
improvements in building efficiency, and other offsets. This gives \gls{uiuc}
the flexibility to continue growing while reducing carbon emissions in other
areas. The breakdown of carbon offsets shown in Figure \ref{fig:icap_emissions}
is improved by adding nuclear power to the energy mix.
Finally, importing electricity drove the campus carbon emissions in every
scenario we examined. If \gls{uiuc} is serious about decarbonizing by 2050, the
University must stop buying electricity from MISO. Unless, that is, energy
production throughout MISO also becomes carbon free.

Besides producing emissions free electricity and steam, nuclear power can
benefit campuses, like \gls{uiuc}, in many ways. Future work will explore how
nuclear power can help decarbonize campus transportation, the role of energy
storage, and peer further into the future.
