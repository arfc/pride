\section{Conclusion}

In this study we used the \gls{esom} called \gls{temoa} to find the optimal
size of a nuclear reactor for the \gls{uiuc} microgrid. We first showed that
\gls{temoa} gave realistic results that matched predictions from both \gls{icap}
and the \gls{uiuc} Master Plan \cite{isee_illinois_2015, affiliated_engineers_inc_utilities_2015}.
Then we considered three scenarios that introduced nuclear capacity to
\gls{uiuc}. The first two scenarios did not constrain the size of the nuclear
reactor and thus satisfied the carbon constraints and exceeded the steam
and electricity demand requirements by building more nuclear capacity than
required.
The \gls{uiuc} Master Plan found that the goals outlined in \gls{icap} could
not be achieved with \gls{uiuc}'s current energy mix, which we corroborated in
our business-as-usual scenario. We showed in Scenario 3 that the \gls{icap}
goals could be met for the next decade by adding a modest capacity for nuclear
energy production.
