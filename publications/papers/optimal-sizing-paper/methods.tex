\section{Methodology}
\gls{temoa} is an open source tool for energy system optimization that
formulates and solves a linear optimization problem
\cite{decarolis_tools_2020}. A linear optimization
problem has two requirements: An objective function and constraints. The
objective function in \gls{temoa} is total system cost over time horizon
of interest and the minimum required constraint is annual demand (and
technology options to meet that demand).

\begin{align}
  \intertext{Minimize}
  C_{sys} &= \sum_{y}^{period} \sum_{t}^{tech} C_{t,y} P_{t,y} F_{t,y} T
  \intertext{Subject to}\\
  M_{y} &= \sum_{t}^{tech} M_{t,y}P_{t,y}F_{t,y}T\\
  \intertext{where}\\
  C_{sys} &= \text{system cost, [\$]}\nonumber\\
  C_{t,y} &= \text{technology cost, } \left[\frac{\$}{\text{MWh}}\right]\nonumber\\
  P_{t,y} &= \text{technology capacity, [MW]}\nonumber\\
  F_{t,y} &= \text{technology capacity factor, [-]}\nonumber\\
  T &= \text{time, [hours]}\nonumber\\
  M_{y} &= \text{emissions for year, $y$, [tons CO$_2$eq]}\nonumber\\
  M_{t,y} &= \text{technology emissions, } \left[\frac{\text{CO}_2\text{eq}}{\text{MWh}}\right]\nonumber
\end{align}

Users can optionally add other
constraints to match the real system being modeled. In our case we added
emissions limits based on the carbon goals set by \gls{icap}. At each time step,
\gls{temoa} must be able to meet the various constraints with the existing
capacity, or build be able to build new capacity to do so. If demand and
emissions limits cannot be satisfied, then \gls{temoa} gives ``no solution.''
A detailed description of \gls{temoa}'s mathematics can be found at
``temoacloud.com.'' For uncertainty analysis, \gls{temoa} implements the \gls{hsj} algorithm for \gls{mga}.
