\section{Introduction}

Load following has given natural gas an economic edge over nuclear power
because unlike nuclear,
natural gas plants can follow grid demand and even shut off when renewable
penetration makes the
price of electricity go negative \cite{keppler_carbon_2011}. Advanced reactor
designs, like Molten Salt
Reactors (MSR), promise strong load following capabilities due to harder
neutron spectra and faster Xe-135 burnup \cite{rykhlevskii_impact_2019}.
Unfortunately, the most mature MSR
designs are at least a decade away from obtaining a commercial license in the
United States. The climate crisis is too urgent to wait this long for nuclear
power to become fully competitive with natural gas.
Renewable energy has also challenged the base load electricity production that
nuclear provides by introducing unpredictable variability in grid demand.
Nuclear energy can be more economically feasible by relaxing the strong load
following requirements
with improved predictions of renewable energy production several hours or days
in advance. If reactor operators knew in advance how much electricity will be
produced by renewable energy they
can slowly and accurately ramp reactor power to meet demand rather than operate
continuously at full power and risk paying to export electricity. Thus
improving nuclear energy's comptetitiveness against natural gas and
strengthening nuclear's ability to couple with renewable energy.
In this work we introduce \acrfull{ESN} as a preferred method for time 
series forecasting of chaotic and stochastic systems like electricity
production from renewable sources.
