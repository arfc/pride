\section{Background}
The University of Illinois at Urbana-Champaign is an ideal model system for
this work because of its diverse energy mix. Previous work has been done to
characterize this energy grid and optimize the size of a nuclear reactor \cite{dotson_optimal_2020}. Due to the degree of wind penetration, the University is sometimes
forced to sell electricity back to MISO at a loss because of overproduction
from wind energy. Thus, a reliable prediction of electricity production from
wind and other variable sources will reduce the likelihood of these events,
specifically. In general, this prediction will also allow nuclear power plants
to bid on the wholesale market, rather than act as price takers.
\acrshort{ESN}s, a flavor of reservoir computing, are a modern
machine learning algorithm that enables accurate short
to medium term predictions. Pathak et. al used an \acrshort{ESN} to predict the
evolution of a chaotic system, a laminar flame front, up to seven Lyapunov
times in the future \cite{pathak_model-free_2018, wikner_combining_2020}. A
Lyapunov time simply measures the timescale at which chaos makes initial
predictions useless. The effect of chaos typically overwhelms conventional
predictions after a single Lyapunov time, by definition.
The Lyapunov time for a weather system is on the order of a few days but
depends on regional environment. \acrshort{ESN}s have also been used to
forecast multivariate time series
\cite{bianchi_reservoir_2020}. Echo state networks are unique among neural
networks in their ease of implementation and training speed. This is owed to its
sparse network architecture \cite{pathak_model-free_2018,
wikner_combining_2020, vannitsem_predictability_2017}. However,
their simplicity is balanced by the need for carefully chosen hyperparameters
for the desired task \cite{lukosevicius_practical_2012}.
Combining accurate demand predictions with reliable
renewable energy predictions will enable an artificially intelligent reactor
operator to adjust power in a relaxed manner.
